\secnumberlesssection{INTRODUCCIÓN}

Los algoritmos de navegación y localización simultánea en la robótica son relativamente nuevos, sin embargo en las últimas dos décadas dado los avances de la tecnología, grandes avances se han producido en el área de la robótica móvil permitiendo algoritmos más avanzados y optimizados.

En la actualidad, existen diversos sensores y algoritmos para enfrentar el problema de la localización de un robot, es por ello que es esencial darle solución a los 4 problemas principales de dichos algoritmos: La asociación de los datos, la incertidumbre, el error acumulativo y la complejidad temporal. Para darle solución a dicho problema en el documento se plantea la implementación de un algoritmo SLAM 3D que permita generar el mapa tridimensional y bidimensional del entorno utilizando una cámara de profundidad y un LIDAR y de esta manera mejorar la asociación de los datos y por ende, generar mapas más precisos a un bajo coste monetario.

El objetivo del documento es la creación de un algoritmo de localización y navegación simultánea, el cual se busca cumplir a través de la investigación del estado de arte de dichos algoritmos, el diseño propiamente tal de dicho algoritmo utilizando solo una cámara de profundidad y un LIDAR y también la implementación del algoritmo en un robot físico y así evaluar en ambientes controlados y reales el desempeño de este.

La estructura del documento presenta 6 capítulos: El capítulo 1 corresponde a la definición del problema en donde se explica el contexto, el objetivo y los impactos de darle solución a la problemática del SLAM en la robótica. Luego en el capítulo 2 se presenta el marco conceptual para entregar los conocimientos técnicos al lector así como también, se describe el estado del arte de los algoritmos SLAM. En el capítulo 3 se plantea la propuesta de solución, en donde se describe el punto de partida y luego el diseño del algoritmo. Luego, en el capítulo 4 se realiza el diseño experimental de la memoria en donde se describe el robot utilizado para la memoria y también se detallan tanto las métricas de evaluación como los 5 ambientes experimentales utilizados. Posteriormente, en el capítulo 5 se detallan los resultados obtenidos en los 5 ambientes, así como también se describe el proceso de calibración de los sensores. Finalmente, en el capítulo 6 se realizan las conclusiones de la memoria y se plantea los lineamientos para un trabajo futuro a quienes deseen continuar con la investigación.

\begin{comment}
Debe proporcionar a un lector los antecedentes suficientes para poder contextualizar en general la situación tratada, a través de una descripción breve del área de trabajo y del tema particular abordado, siendo bueno especificar la naturaleza y alcance del problema; así como describir el tipo de propuesta de solución que se realiza, esbozar la metodología a ser empleada e introducir a la estructura del documento mismo de la memoria.

En el fondo, que el lector al leer la Introducción pueda tener una síntesis de cómo fue desarrollada la memoria, a diferencia del Resumen dónde se explicita más qué se hizo, no cómo se hizo.

\end{comment}