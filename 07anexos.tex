\secnumberlesssection{ANEXOS}

\begin{comment}

\textbf{PREPARACIÓN DEL AMBIENTE DE LA MEMORIA}

\begin{verbatim}
    sudo sh -c 'echo "deb http://packages.ros.org/ros/ubuntu 
    $(lsb_release -sc) main" > /etc/apt/sources.list.d/ros-latest.list'
\end{verbatim}

\begin{verbatim}
    sudo apt install curl # if you haven't already installed curl
    curl -s https://raw.githubusercontent.com/ros/rosdistro/master/ros.asc 
    | sudo apt-key add -
\end{verbatim}

\begin{verbatim}
    sudo apt update
\end{verbatim}

\begin{verbatim}
    sudo apt install ros-noetic-desktop-full
\end{verbatim}

\begin{verbatim}
    echo "source /opt/ros/noetic/setup.bash" >> ~/.bashrc
    source ~/.bashrc
\end{verbatim}

\begin{verbatim}
    sudo apt install python3-rosdep python3-rosinstall 
    python3-rosinstall-generator python3-wstool build-essential
\end{verbatim}

\begin{verbatim}
    sudo apt install python3-rosdep
\end{verbatim}

\begin{verbatim}
    sudo rosdep init
    rosdep update
\end{verbatim}

\newpage
\textbf{SECCIÓN DE CÓDIGOS DEL TIPO LAUNCH FILE}

\textbf{ros\_serial.launch}
\begin{verbatim}
<launch>
    <node 
        pkg="rosserial_python" 
        name="arduino_node" 
        type="serial_node.py">
        <param 
            name="port" 
            value="/dev/ttyACM0" 
        />
        <param 
            name="baud" 
            value="115200" 
        />
    </node>
</launch>
\end{verbatim}

\textbf{ld19.launch}
\begin{verbatim}
<launch>
    <node 
        name="LD19" 
        pkg="ldlidar_stl_ros" 
        type="ldlidar_stl_ros_node">
        <param 
            name="product_name" 
            value="LDLiDAR_LD19"/>
        <param 
            name="topic_name" 
            value="scan"/>
        <param 
            name="port_name" 
            value ="/dev/ttyUSB0"/>
        <param 
            name="frame_id" 
            value="base_laser"/>
        <param 
            name="laser_scan_dir" 
            type="bool" 
            value="true"/>
        <param 
            name="enable_angle_crop_func" 
            type="bool" 
            value="false"/>
        <param 
            name="angle_crop_min" 
            type="double" 
            value="135.0"/>
        <param 
            name="angle_crop_max" 
            type="double" 
            value="225.0"/>
    </node>
    <node 
        name="base_to_laser" 
        pkg="tf" 
        type="static_transform_publisher"  
        args="0.0 0.0 0.18 4.71 0.0 0.0 base_link base_laser 50"/>
</launch>
\end{verbatim}

\textbf{gazebo.launch}
\begin{verbatim}
<launch>
    <arg 
        name="extra_gazebo_args" 
        default="--verbose"/>
    
    <arg 
        name="world_file" 
        default="$(
            find robocop_description)/
            worlds/
            simulation_world_00/
            simulation_world_00.world"/>
    
    <param 
        name="robot_description" 
        command="$(find xacro)/xacro 
        $(find robocop_description)/urdf/robocop.xacro"/>
    <node 
        name="spawn_urdf"
        pkg="gazebo_ros" 
        type="spawn_model" 
        args="-param robot_description -urdf -model robocop"/>
    <include 
        file="$(find gazebo_ros)/launch/empty_world.launch">
        <arg 
            name="world_name" 
            value="$(arg world_file)"/>
        <arg 
            name="paused" 
            value="false"/>
        <arg 
            name="use_sim_time" 
            value="true"/>
        <arg 
            name="gui" 
            value="true"/>
        <arg 
            name="headless" 
            value="false"/>
        <arg 
            name="debug" 
            value="false"/>
        <arg 
            name="extra_gazebo_args" 
            value="$(arg extra_gazebo_args)"/>
    </include>
</launch>
\end{verbatim}

\textbf{camera.launch}
\begin{verbatim}
<launch>
    <node 
        name="usb_cam" 
        pkg="usb_cam" 
        type="usb_cam_node">
        <param 
            name="video_device" 
            value="/dev/video2" />
        <param 
            name="image_width" 
            value="640"/>
        <param 
            name="image_height" 
            value="480"/>
        <param 
            name="pyxel_format" 
            value="yuvj420p"/>
        <param 
            name="image_format" 
            value="rgb8"/>
        <param 
            name="camera_frame_id" 
            value="usb_cam"/>
        <param 
            name="io_method" 
            value="mmap"/>
    </node>
    <node 
        name="image_view" 
        pkg="image_view" 
        type="image_view" 
        respawn="false">
        <remap 
            from="image" 
            to="usb_cam/image_raw"/>
        <param 
            name="autosize" 
            value="true"/>
    </node>
</launch>
\end{verbatim}

\textbf{robocop\_map.launch}
\begin{verbatim}
<launch>
    <include 
        file="$(find robocop_description)/launch/ld19.launch"/>
    <include 
        file="$(find robocop_description)/launch/mapping/display_map.launch"/>
    <include 
        file="$(find robocop_description)/launch/mapping/ps4.launch"/>
    <include 
        file="$(find robocop_description)/launch/mapping/hector_mapping.launch"/>
</launch>
\end{verbatim}

\textbf{ps4.launch}
\begin{verbatim}
<launch>
    <group ns="j1">
        <node 
            name="ds4_joystick" 
            pkg="joy" 
            type="joy_node">
                <param 
                    name="dev" 
                    value="/dev/input/js1" />
        </node>
    </group>
    <node 
        pkg="robocop_description" 
        name="ps4Controller"
        type="ps4Controller.py"/> 
    <node 
        pkg="rosserial_python" 
        name="arduino_node" 
        type="serial_node.py">
        <param 
            name="port" 
            value="/dev/ttyACM0" />
        <param 
            name="baud" 
            value="115200" />
    </node>
</launch>
\end{verbatim}

\textbf{hector\_mapping.launch}
\begin{verbatim}
<launch>
    <node 
        pkg="hector_mapping" 
        type="hector_mapping" 
        name="hector_mapping">
        <param 
            name="base_frame" 
            value="base_link" />
        <param 
            name="odom_frame"
            value="base_link" />
    </node>
</launch>
\end{verbatim}

\textbf{display\_map.launch}
\begin{verbatim}
<launch>
    <arg 
        name="model" 
        default="$(find robocop_description)/urdf/robocop.xacro"/>
    <arg 
        name="multi_robot_name" 
        default=""/>
    
    <param 
        name="robot_description" 
        command="$(find xacro)/xacro $(arg model)"/>
    <node 
        pkg="rviz" 
        type="rviz" name="rviz" 
        required="true" 
        args="-d $(find robocop_description)/launch/robocop.rviz"/>

    <node 
        pkg="robot_state_publisher" 
        type="robot_state_publisher" 
        name="robot_state_publisher">
        <param 
            name="publish_frequency" 
            type="double" 
            value="50.0"/>
        <param 
            name="tf_prefix" 
            value="$(arg multi_robot_name)"/>
    </node>
    <node 
        name="joint_state_publisher" 
        pkg="joint_state_publisher" 
        type="joint_state_publisher"/>
</launch>

\end{verbatim}

\textbf{robocop\_auto.launch}
\begin{verbatim}
<launch>
    <include 
        file="$(find robocop_description)/launch/ld19.launch"/>
    <include 
        file="$(find robocop_description)/launch/auto/display_auto.launch"/>
    <include 
        file="$(find robocop_description)/launch/ros_serial.launch"/>
</launch>

\end{verbatim}

\textbf{display\_auto.launch}
\begin{verbatim}
<launch>
    <arg 
        name="model"
        default="$(find robocop_description)/urdf/robocop.xacro"/>
    <arg 
        name="multi_robot_name" 
        default=""/>
    <param 
        name="robot_description"
        command="$(find xacro)/xacro $(arg model)"/>
    <node 
        pkg="rviz" type="rviz" 
        name="rviz" required="true" 
        args="-d $(find robocop_description)/launch/robocop.rviz"/>
    
    <node 
        pkg="robot_state_publisher" 
        type="robot_state_publisher" 
        name="robot_state_publisher">
        <param 
            name="publish_frequency" 
            type="double" 
            value="50.0" />
        <param 
            name="tf_prefix" 
            value="$(arg multi_robot_name)"/>
    </node>
    <node 
        name="joint_state_publisher" 
        pkg="joint_state_publisher" 
        type="joint_state_publisher"/>
    <node 
        name="map_server" 
        pkg="map_server" 
        type="map_server" 
        args="$(
            find robocop_description)/
            worlds/
            real_world_00/
            real_world_00.yaml" />
    <include 
        file="$(
            find robocop_description)/
            launch/
            auto/
            amcl_robocop.launch"/>
    
    <node 
        pkg="move_base" 
        type="move_base" 
        respawn="false"
        name="move_base" 
        output="screen">
        <param 
            name="base_local_planner" 
            value="dwa_local_planner/DWAPlannerROS" />
    <rosparam 
        file="$(
            find robocop_description)/
            codes/
            navigation_stack/
            costmap_common_params.yaml" 
        command="load" 
        ns="global_costmap" />
    <rosparam 
        file="$(
            find robocop_description)/
            codes/
            navigation_stack/
            costmap_common_params.yaml" 
        command="load" 
        ns="local_costmap" />
    <rosparam 
        file="$(
            find robocop_description)/
            codes/
            navigation_stack/
            local_costmap_params.yaml" 
        command="load" />
    <rosparam 
        file="$(
            find robocop_description)/
            codes/
            navigation_stack/
            global_costmap_params.yaml" 
        command="load" />
    <rosparam 
        file="$(
            find robocop_description)/
            codes/
            navigation_stack/
            move_base_params.yaml" 
        command="load" />
    <rosparam file="$(
        find robocop_description)/
        codes/
        navigation_stack/
        dwa_local_planner_params.yaml" 
        command="load" />
    </node>
</launch>


\end{verbatim}

\textbf{amcl\_robocop.launch}
\begin{verbatim}
<launch>
    <node 
        pkg="amcl"
        type="amcl" 
        name="amcl">
        <param 
            name="min_particles"             
            value="500"/>
        <param 
            name="max_particles"             
            value="3000"/>
        <param 
            name="kld_err"                   
            value="0.02"/>
        <param 
            name="update_min_d"              
            value="0.20"/>
        <param 
            name="update_min_a"              
            value="0.20"/>
        <param 
            name="resample_interval"         
            value="1"/>
        <param 
            name="transform_tolerance"      
            value="0.5"/>
        <param 
            name="recovery_alpha_slow"      
            value="0.00"/>
        <param
            name="recovery_alpha_fast"      
            value="0.00"/>
        <param 
            name="gui_publish_rate"          
            value="50.0"/>
        <param 
            name="laser_max_range"           
            value="8"/>
        <param 
            name="laser_min_range"           
            value="0.3"/>
        <param 
            name="laser_max_beams"          
            value="180"/>
        <param 
            name="laser_z_hit"               
            value="0.5"/>
        <param 
            name="laser_z_short"             
            value="0.05"/>
        <param 
            name="laser_z_max"              
            value="0.05"/>
        <param 
            name="laser_z_rand"              
            value="0.5"/>
        <param 
            name="laser_sigma_hit"           
            value="0.2"/>
        <param 
            name="laser_lambda_short"        
            value="0.1"/>
        <param 
            name="laser_likelihood_max_dist" 
            value="2.0"/>
        <param 
            name="laser_model_type"          
            value="likelihood_field"/>
        <param 
            name="odom_model_type"           
            value="diff"/>
        <param 
            name="odom_alpha1"               
            value="0.1"/>
        <param 
            name="odom_alpha2"              
            value="0.1"/>
        <param 
            name="odom_alpha3"               
            value="0.1"/>
        <param 
            name="odom_alpha4"              
            value="0.1"/>
        <param 
            name="odom_frame_id"             
            value="odom"/>
        <param 
            name="base_frame_id"             
            value="base_link"/>
        
        </node>
</launch>
\end{verbatim}

\newpage
\textbf{SECCIÓN DE CÓDIGOS PARA EL CONTROL}

\textbf{ps4Controller.py}
\begin{verbatim}
#!/usr/bin/env python

from sensor_msgs.msg import Joy
import sys, rospy
from geometry_msgs.msg import Twist

def chatter_callback(mensaje):
    joystick_izquierdo = (
        round(mensaje.axes[0],
        3), 
        round(mensaje.axes[1], 3)
        )
    joystick_derecho = (
        round(mensaje.axes[2], 
        3), 
        round(mensaje.axes[5], 3)
        )
    botones = (
        mensaje.buttons[1], 
        mensaje.buttons[2], 
        mensaje.buttons[3], 
        mensaje.buttons[0])

    # Publish data to cmd_vel topic
    publisherMotores = rospy.Publisher('/cmd_vel', Twist, queue_size=10)
    twist = Twist()
    twist.linear.x = joystick_izquierdo[1]
    twist.angular.z = joystick_derecho[0]
    publisherMotores.publish(twist)


def nodoPS4():
    nombreNodo = "nodo_ps4"
    idUnico = True
    rospy.init_node(nombreNodo, anonymous = idUnico)
 
    nombreTopico = "/j1/joy"
    tipoTopico = Joy 

    rospy.Subscriber(nombreTopico, tipoTopico, chatter_callback)
    rospy.spin()


if __name__ == "__main__":
    global arduino
    nodoPS4()
    
\end{verbatim}

\newpage
\textbf{SECCIÓN DE CÓDIGOS PARA LA CÁMARA}

\textbf{camera.py}
\begin{verbatim}
#!/usr/bin/env python

import cv2
import numpy as np
import time

# Create a VideoCapture object
cap = cv2.VideoCapture(2)

# Check if camera opened successfully
if (cap.isOpened()== False):
    print("Error opening video stream or file")

# Read until video is completed
while(cap.isOpened()):
    # Capture frame-by-frame
    ret, frame = cap.read()
    if ret == True:
        # Change the display size
        frame = cv2.resize(frame, (320, 640))
        # Display the resulting frame
        cv2.imshow('Frame',frame)

        # Press Q on keyboard to  exit
        if cv2.waitKey(25) & 0xFF == ord('q'):
            break

    # Break the loop
    else:
        break

# When everything done, release the video capture object
cap.release()
\end{verbatim}

\newpage
\textbf{SECCIÓN DE CÓDIGOS PARA LOS MOTORES}

\textbf{controller\_motor.ino}
\begin{verbatim}
// Se importan las librerías
#include <ros.h>
#include <AFMotor.h>

// Se importan los mensajes
#include <geometry_msgs/Twist.h>

// Variables globales de ROS
ros::NodeHandle nHandler;

// Se instancian las variables
float linearVelocity;
float angularVelocity;
float turnRightVel;
float turnLeftVel;

int forwardRight, forwardLeft, turnRight, turnLeft;
AF_DCMotor Motor1(3);
AF_DCMotor Motor2(4);

const byte encoder_rightA = 18;
const byte encoder_rightB = 19;
const byte encoder_leftA = 20;
const byte encoder_leftB = 21;

static long counter_right = 0;
static long counter_left = 0;
static long counter_right_old = 0;
static long counter_left_old = 0;

volatile bool fired_right, fired_left;
volatile bool up_right, up_left;

double x = 0;
double y = 0;
double theta = 0;

double rightVelocity = 0.0;
double leftVelocity = 0.0;

int long currentTime = 0;
int long previousTime = 0;

float map_float(
    float x, 
    float in_min, 
    float in_max, 
    float out_min, 
    float out_max) {
  return (x - in_min) * (out_max - out_min) / (in_max - in_min) + out_min;
}

void callback(const geometry_msgs::Twist& vel){
  linearVelocity = vel.linear.x;
  angularVelocity = vel.angular.z;

  int right_velocity = 150;
  int left_velocity = 150;
  if (linearVelocity>0.1){
    Motor1.run(FORWARD);
    Motor2.run(FORWARD);
    Motor1.setSpeed(right_velocity);
    Motor2.setSpeed(left_velocity);
  }
  if (linearVelocity<-0.1){
    Motor1.run(BACKWARD);
    Motor2.run(BACKWARD);
    Motor1.setSpeed(right_velocity);
    Motor2.setSpeed(left_velocity);
  }
  if (angularVelocity>0.1){
    Motor1.run(FORWARD);
    Motor2.run(BACKWARD);
    Motor1.setSpeed(right_velocity);
    Motor2.setSpeed(left_velocity);
  }
  if (angularVelocity<-0.1){
    Motor1.run(BACKWARD);
    Motor2.run(FORWARD);
    Motor1.setSpeed(right_velocity);
    Motor2.setSpeed(left_velocity);
  }
  if (abs(angularVelocity)<0.1 and abs(linearVelocity)<0.1){
    Motor1.setSpeed(0);
    Motor2.setSpeed(0);
  }

}

// Creación de los nodos de ROS
ros::Subscriber<geometry_msgs::Twist> nodo_vel("cmd_vel", callback);

// Variables para la odometría
double odom_x = 1.0;
double odom_y = 0.0;
double odom_thet = 1.57;
void setup(){
  // Se ajustan los nodos
  nHandler.getHardware()->setBaud(115200);
  nHandler.initNode();
  nHandler.subscribe(nodo_vel);

  // Se ajustan los motores
  Motor1.setSpeed(0);
  Motor2.setSpeed(0);
  Motor1.run(RELEASE);
  Motor2.run(RELEASE);


  Serial.begin(115200);
}

void loop(){ 
  nHandler.spinOnce();

}

\end{verbatim}

\end{comment}